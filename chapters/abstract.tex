Maritime surveillance demands a broad range of technological systems that improve the overall situational awareness in order to ensure safety and security. To this extent, an ongoing research topic is the construction of a sophisticated method that is capable of forecasting accurate end-to-end vessel trajectories purely utilizing historical AIS signals. Due to the recent achievements of deep reinforcement learning and the mimicking nature of behavioral cloning which suits the task flawlessly, this thesis investigates whether one of those methods is a feasible approach to predict individual ship paths.
\par
A profound literature research has shown that related work mainly focuses on autonomous ship navigation that tries to optimize ship routes and controls, neglecting or restricting the task of anticipating precise vessel trajectories. 
\par
After a very deep examination of the theoretical backgrounds of the used algorithms, namely deep deterministic policy gradient (DDPG) and behavioral cloning, we have created a synthetic, simplified experimental setup of predicting just three distinct curves. The results of those experiments reveal, that DDPG fails catastrophically while behavioral cloning performs almost perfectly in following and forecasting the underlying ground-truth curves. \par
Based on the experience and findings that have been gathered during the synthetic experiments, we have applied behavioral cloning to the custom-built environment that uses historical AIS data from the coast and port of Bremerhaven. We have described the whole procedure of preprocessing, filtering, and eventually grouping the data, leading to four datasets of different ship categories such as tankers or cargo ships. Simultaneously, we have defined different state representations and action spaces, as well as two very distinct underlying transition dynamics. The initial poor results have been significantly improved after we have recognized that crucial information about the destination is missing. After making the system Markov and the agent able to distinguish between different motion clusters, the best model has achieved a median error between the prediction and the ground-truth of 433 meters. This is a very promising result, which is close to our defined threshold of 300 meters as the desired goal.
\par
Besides its propitious accuracy in predicting vessel trajectories from an arbitrary number of motion clusters, the presented system further convinces with its low computational costs, as well as its flexibility to changes of the feature space. Additional improvements, which have also been discussed in this thesis, could further improve the system to make it a viable option for the real-world use.