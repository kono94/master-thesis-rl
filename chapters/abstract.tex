Maritime surveillance demands a broad range of technological systems that improve the overall situational awareness in order to ensure safety and security. To this extend, an ongoing research topic is the construction of a sophisticated method that is capable of forecasting accurate end-to-end vessel trajectories purely utilizing historical AIS signals. Due to the recent achievements of deep reinforcement learning and the mimicking nature of behavioral cloning which suits the task flawlessly, this thesis investigates whether one of those methods is a feasible approach to predict individual ship paths.
\par
The profound literature research shows that related work mainly focuses on autonomous ship navigation that tries to optimize ship routes and controls, neglecting or restricting the task of anticipating precise vessel trajectories. 
\par
After a very deep examination of the theoretical backgrounds of the used algorithms, namely deep deterministic policy gradient (DDPG) and behavioral cloning, we have created a synthetic, simplified experimental setup of predicting just three distinct curves. The results of those experiments reveal, that DDPG fails catastrophically while behavioral cloning performs almost perfectly in following and forecasting the underlying ground-truth curves. \par
Based on the experience and findings that have been gathered during the synthetic experiments, we apply behavioral cloning to the custom built environment that uses historical AIS data from the coast and port of Bremerhaven. We describe the whole procedure of preprocessing, filtering, and eventually grouping the data, leading to four datasets of different ship categories such as tankers or cargo ships. Simultaneously, we have defined deviating state representations and action spaces, as well as two very distinct underlying transition dynamics. Although behavioral cloning has performed well in the synthetic setup, the conducted experiments in the AIS scenario show a significant decline in performances with an average distance between the prediction and the true position of more than 2500 meters (for the dataset including ships categorized as tanker).
\par
Analyses regarding different neuronal network architectures demonstrate the fragility of behavioral cloning. In spite of the fact that the overall task of building an accurate predicting system has been unsuccessful, this thesis has presented important rudiments, approaches, pitfalls, and ideas for further improvements.