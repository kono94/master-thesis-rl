To our best knowledge there is no published work that takes vessel path prediction for arbitrary ship models or the generation of representative vessel trajectories based on historical AIS data into the realm of reinforcement- or imitation learning. As investigation in chapter \ref{chap:relatedWork} will show, related work is focused on autonomous control with the goal of unmanned ship navigating to a desired destination. Hence, a major part of the motivation is the study of a novel approach to the path prediction problem in the maritime domain based on reinforcement- and imitation learning to a flexible set of different ship types.
\par
Moreover, motivation arises from the huge interest of the author for the field of reinforcement learning. In his previously prepared bachelor thesis, the author has already intensively dealt with the absolute basics of reinforcement learning.  However, the investigated approaches were limited to tabular methods, which are a big restriction to framework in that they are not suitable for complex problems with vast state and action spaces. The examination of deep reinforcement learning and the associated use of neural networks as powerful function approximation is a logical next step for the author to further educate himself in this area.
\par
In addition, an extended abstract to this work was submitted for the MARESEC 2022 conference. This abstract was prepared in the middle phase of this thesis, with the plan to further refine the results listed there by the date of the conference in summer. The complete submitted abstract can be viewed under section \ref{appendix:maresec} in the appendix.