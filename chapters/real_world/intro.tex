This chapter presents the setup, performance and weakness of a system that should learn representative vessel trajectories in order to generate and predict future paths. The overall system is built based on the findings of the synthetic experiments which were deeply discussed in the previous chapter \ref{chap:synthetic}. Since the conclusion of the last chapter is that DDPG performs badly even in a simplified environment and is therefore not applicable to the overall problem defined by this thesis, we exclusively concentrate on the usage of behavioral cloning as primary method. 
\par
First, the process of handling historical AIS data and extracting ship trajectories is described. The resulting dataset acts as ground truth (GT) in the custom environment which is explained in the next subchapter. Afterwards, we use the same generic metric of average euclidian distances as defined by Eq. \ref{eq:euclid} to quantify the performance of the system and to discuss the results. Finally, we trace and analyse supposedly bad predictions to potentially discover patterns which lead to poor performances.