This chapter presents the setup, performance, and weaknesses of a system that should learn representative vessel trajectories in order to generate and predict future paths. The overall system is built based on the findings of the synthetic experiments, which were deeply discussed in the previous chapter \ref{chap:synthetic}. Since the conclusion of the last chapter is that DDPG performs badly even in a simplified environment and is therefore not applicable to the overall problem defined by this thesis, we exclusively concentrate on the usage of behavioral cloning as primary method. 
\par
First, the process of handling historical AIS data and extracting ship trajectories is described. The resulting dataset acts as ground truth (GT) in the custom environment, which is explained in the next subchapter. Afterwards, we use the metric of average euclidean distances as defined by Eq. \ref{eq:euclid} per episode, but quantify the performance of the system by taking the median values, as well as the absolute deviation from the median. After noticing very poor performances in the first set of experiments in subchapter \ref{subchap:aisResults}, we will expand the state representation by adding computed features regarding the destination of vessels, which in return greatly boosts the accuracy of predictions.
\par
Before talking about "good" and "bad" performances and predictions during the span of this chapter, we first have to defined what is an acceptable error tolerance for the system of forecasting future vessel paths. To our best knowledge, there are no previously established thresholds in the literature, so we have to defined our own. One of the most important aspects, in this regard, is the correct interpretation of the present dimensions. The observation window is approximately 362 square kilometers, cargo ships can easily be 200 meters long (e.g., the randomly taken bulk carrier "FEDERAL LEDA" with MMSI 538005611 is 200 meters in length and 24 meters in width), and the general uncertainty of the AIS signal itself (10 meters, see \ref{chap:ais}). Here, the lengths of ships can be directly tied to the acceptance of deviating position between the prediction and the ground-truth. What we mean by this is, that if a predicted position 10 minutes ahead of time for a 200-meter vessel is off by 300 meters, it could indeed be labeled as a "good" estimation. However, the same prediction error for a 20-meter ship is worse and might not be accurate enough to be meaningful for a system in use for maritime surveillance. This fact, in combination with the pure size of the observation space and the inherent error of the AIS signal, lead us to the defined value for the threshold of an "acceptable" prediction of 300 meters in average euclidean distance from the predicted path to the ground-truth.