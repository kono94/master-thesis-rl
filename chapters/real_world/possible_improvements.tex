Due to time constraints, it is not possible to test every single hypothesis and assumption. However, we will share our experience, uncertainties, and ideas in this subchapter to make them visible for scientists dealing with the same kind of topic.
\par
To start with, we would like to focus on the specific values of state and action space. The input features consist of values derived from the AIS signal, e.g., latitude and longitude. One problem with those values is that just small changes in the fractional digits result in big changes to the true position. Even though, we scaled the values for latitude and longitude to 0-1, the observation space might be too big (25 km $\times$ 14.5 km) and thus causes inaccurate predictions purely based on fractional digits. One suggestion is that multiple models should be trained that are responsible for much smaller areas, leading to bigger differences in decimal places of the input features. Also, the use of a "Standardscaler" could be beneficial, which causes the individual features to have a mean of 0 and a standard deviation of 1.
\par
Another assumption is that the features space does not cover up all potential influences to the controls taken by the captain. If we rightfully assume that the current heavily influences the motion patterns and controls of ships, then excluding this information from the feature space makes it impossible for the agent to learn an optimal policy. The underlying causality is missing. A possible next step is the extension of the state space with features like wind condition and tide information.
\par
As we could see in subchapter \ref{subchap:importance}, the results were significantly improved by adding information about the destination to the feature space. This change made the environment Markov in the first place, allowing the agent to distinguish between different motion clusters. Another approach could be the direct separating of different clusters and the usage of one model per cluster. The task of each model is then reduced to learn the differences in behavior within a certain cluster, without the need to determine the corresponding cluster first. This should greatly boost the performance.
\par
Last but not least, the usage of different network models, e.g., transformers or generative adversarial networks could be tested as well. The elementary approach of behavioral cloning simply makes use of a fully connected network.