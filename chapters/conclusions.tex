During the process of crawling through related literature and the writing of the corresponding chapter \ref{chap:relatedWork}, we found out that related work predominately focuses on methods that can be best described as "pathfinders", meaning that they try to optimize navigation or are part of the field of autonomous control. Besides, publications in the same scope of this thesis, namely the forecasting of vessel paths for anomaly detection, are too restrictive in terms of, e.g., just using a few ship instances or similar trajectories belonging to the same cluster.
\par
The frequent mentions of an algorithm called deep deterministic policy gradient (DDPG) led to a deep examination of its internal workings and the influence of certain hyperparameters. Unfortunately, the conducted synthetic experiments show that DDPG is unable to reproduce three simplistic curves. Increasing the replay buffer size does not change the outcome. Having a $\sigma$-value greater than 0.3 impacts the performance negatively, which is inline with the standard parameters used in the literature and programming frameworks. Moreover, the theoretical assumption made in terms of different learning rates holds because a higher $\alpha$-value for the critic in contrast to the actor does indeed yield better results. Also, increasing the amount of features, e.g., by adding the current distance to the ground-truth, enhances the ability of the agent to distinguish between varying environmental snapshots at different time steps across all hyperparameter setups. Nevertheless, the performances of separate runs fluctuate significantly, making DDPG unsuitable for the general task. We also conclude that a method needs to be specialized in inherent learning from historical decisions.
\par
This approach is best achieved by imitation learning. In this regard, we choose behavioral cloning by describing the parallels to biology and nature. When being applied to the same task of reproducing three arbitrary curves, without the need for generalization, behavioral cloning performs exceptionally well.
\par
In preparation of applying behavioral cloning to the custom-built environment which uses real-world AIS signals, the dataset of trajectories derived from those historical AIS records is an important aspect.
We took the route of cleaning the data enormously, making the extracted trajectories as clear and smooth as possible, without any noticeable outliers but with almost exclusively long trajectories of more than 1500 meters. Additionally, we assembled multiple dataset dependent on the ship types. For example, we removed ships like tugs or sailing boats for their distinct movement patterns, building the "big ships"-dataset and constructed additional dataset that only include ships that can be classified as "cargo ships" or "tankers", resulting in four different datasets.
\newpage
At first, the initial set of experiments and resulting policies, when applying behavioral cloning to the problem, do not yield a performance that can be labeled as accurate. Precisely, the prediction error, meaning the average distance between the agent's generated path and the ground-truth, is more than 2000 meters. A closer look at the median values, which are less bias than the average, revealed that the "big ships"-dataset is best suited for training because it excludes ships that have totally different motion patterns than the rest (e.g., tugs and sailing ships) but also keeps a more workable size of training data in contrast to the "tanker"- and "cargo"-datasets. Even deeper analyses and the development of experiments with varying time intervals between actions and training epochs, different network architectures, or state and action spaces, did not boost the performance to any meaningful level. 
\par
However, after including information about the destination, namely the angle and distance from the current position of the agent to the respective destination, the model produces promising results with a median prediction error of 433 meters and an average euclidean distance of 891 meters. The addition of destination to the state representation made the environment Markov, meaning that the agent is able to fully distinguish between different trajectories and clusters purely based on the initial starting state $s_0$. Although the results can not be labeled as "good" based on our own definition and the threshold of 300 meters, the predictions and generated paths are impressive, especially when having in mind that the "big ships"-dataset consists of more than 1000 unique ships, different ship types and numerous motion clusters.
\par
Regarding the definition of requirements in subchapter \ref{subchap:objective}, behavioral cloning fulfills the demands in flexibility and computation cost, as it allows for the simple addition of features to the state space, as well as single core performance of approx. 8 end-to-end trajectory predictions per second, that is easily scalable to a multicore setup.
\par
Overall, the conducted experiments and results suggest that behavioral cloning with its underlying reinforcement learning framework is a feasible method to construct a system that forecasts vessel trajectories. Further research towards potential improvements like the addition of non-kinematic features to the state representation, the fragmentation of the observation window, and supplementary mechanisms when using neuronal network, could boost the performance in a way to make the system usable in a real-world scenario.