This subchapter explains the underlying communication system that produces the historical ship trajectory data which in return acts as ground truth during learning representative vessel paths. Giving a broad overview of the standard, this chapter is meant for readers that are not familiar with the domain of maritime traffic.
\par
To ensure and enhance maritime safety, ships are constantly broadcasting information about their current position, speed, course etc. At the same time they are receiving the broadcasts from all other ships around them, resulting in a superior understanding of the situation in order to e.g. avoid collisions. 
\par
The whole system behind exchanging those ship information is summarized under the term \textit{Automatic Identification System} (AIS). In 2000, the International Maritime Organization (IMO) which is a specialised agency of the United Nations responsible for regulating shipping, made it mandatory for a list of ship types to carry an AIS \cite[]{imo}. This list includes ships 
with more than 300 gross tonnage (international voyages), cargo ships with more than 500 gross tonnage and all passenger ships irrespective of size \cite[]{imo}.
\par
All intentions, specifications and regulations regarding AIS are documented in the official and revised "Guidelines for the Onboard Operational Use Of
Shipborne Automatic Identification Systems (AIS)" \cite[]{international2015revised}. These guidelines also mention the objectives for AIS to be the:
\begin{itemize}
    \item identification of ships and assistance in target tracking
    \item simplification of the information exchange (e.g. reduce verbal mandatory ship reporting)
    \item help of port authorities or other shore-based facilities to enhance situation awareness
    \item assistance in search and rescue operations
    \item avoidance of ship collisions
\end{itemize}

The content of an AIS message depends on the type of the information. There are three different types that can be transmitted, specifically static-, dynamic- and voyage-related information. In this context, static information are defined per vessel and include the \textit{Maritime Mobile Service Identity} (MMSI) which is unique for every ship as well as other ship dependent information such as length and type of ship. A list of all ship types can be found in the official installation guide of AIS published by \cite{imo2}. In this thesis we will mainly focus on ship types that have a 6 (passenger), 7 (cargo) or 8 (cargo) as first digit while ignoring type 52 (tugs) because of their very distinct maneuvering.
\par
Voyage-related information should in theory consist of the ship's draught, the cargo type, the destination and ETA as well as a route plan. However, those data points have to be entered manually and thus are suffering from high variance not only in availability but also credibility. Although, the knowledge about the destination could be a valuable attribute, especially if it comes down to the distinction of certain vessel paths, we will ignore all voyage-related information.
\par
Besides the static- and voyage-related information which is broadcasted every 6 minutes or upon request, the dynamic information mainly consist of kinematic data that gets send out every 2 to 10 seconds depending on the ship type and the current speed and course alteration. If a ship is at anchor, the reporting interval is extended to 3 minutes. The content of dynamic AIS message includes the position given by latitude and longitude, a UTC timestamp, speed over ground (SOG) and course over ground (COG), the heading of the ship, the current navigational status (underway by engines, at anchor, engaged in fishing etc.) and the rate of turn (ROT). 
\par
The values for all of those data points are collected from ship sensors that are connected to the AIS. Therefore, the IMO warns that "poorly configured or calibrated ship sensors (position,
speed and heading sensors) might lead to incorrect information being transmitted" \cite[p.~11]{international2015revised}. They also state that the accuracy indication of the position is approximately 10 meters (p.~6).