When we do write "learn like human" like in the preceded subchapter, we refer to \textit{the} underlying motivation of reinforcement learning and the fact that it draws inspiration from psychological learning theories \cite[p.~341]{Sutton1998} and neuroscience \cite[p.~377]{Sutton1998}. Reinforcement learning is heavily influenced by the way humans and animals learn and interact with their real-world environment, based on for example chemical reward signals such as dopamine \cite[p.~383]{Sutton1998}. Taking nature or biology as blueprint to build artificial intelligence is not exclusive to the rl-framework. Neuronal networks for instance try to essentially copy the way our human brain works, with components even named "neurons" which are connected to each other (like synapses do in our brain) that reach different levels of activity (activation function).
\par
In our opinion it is a great advantage of reinforcement learning in general that it is so closely related to psychology and nature as a whole. It allows the researcher to take a step back from the formalism, algorithms and programming part in an effort to look at the problem from different perspective. 