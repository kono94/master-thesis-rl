Five month is a short period of time for such a complex research project. Therefore we try to avoid pitfalls by taking into account the advice of our supervisors as well as a great experience report on short-term machine learning projects in context of a master thesis from \cite{tim2018}. He is an associate professor at the University College London and a scientist at the Facebook AI Research organization. The main advice he is giving to students is that they should start from  abstract and simplistic scenarios so it is easier to fully understand what is going on, to then have the ability to intervene and change things up under clear assumptions.
\par 
The planned sequence of tasks involve:
\begin{itemize}
    \item An in-depth literature review to discover similar works and recent trends
    \item The implementation of at least one deep reinforcement learning algorithm from scratch using PyTorch (most likely starting with DDPG)
    \item Designing a simple environment to show that reinforcement learning (RL) is indeed able to learn to reconstruct a family of curves (proof-of-work) 
    \item Applying the RL algorithm to the same learning setting (involving AIS data) of the previous approaches at the institute and compare results
    \item Tweaking of state representation and increase the number of AIS vessel trajectories
    \item (optional, plan B) The analysis of other methods such as offline RL and the usage of the transformer architecture as proposed by \cite{chen2021decision} 
\end{itemize}