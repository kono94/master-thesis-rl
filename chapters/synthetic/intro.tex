Instead of diving directly into the full task of building a sophisticated environment that handles real-world AIS data and afterwards somewhat mindlessly apply algorithms to it, we first try to collect experience by setting up synthetic experiments. Those preparatory experiments are way more simplistic and encourage for better analysis regarding the behavior and performance of certain algorithms, the modelling of the reward function as well as the overall approach to the task itself. Please note that major parts of the proposed setup in this chapter and the findings are the foundation of the actual path prediction system.
\par
This chapter introduces the reader to the experimental environment called "Curve". Furthermore, it sums up the process we go through when it comes to tweaking the reward function or state representation, the reasons that lead to the use of imitation learning as the primary method and the choice of external code libraries.